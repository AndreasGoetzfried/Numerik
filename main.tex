\documentclass{article}
\usepackage{graphicx}
\usepackage[utf8]{inputenc}
\usepackage{listings}
\usepackage{color}
\usepackage{amsfonts}
\usepackage{tabularx}
\usepackage{mathtools}
\usepackage[ngerman]{babel}
\newcommand\norm[1]{\left\lVert#1\right\rVert}

\begin{document}
\begin{titlepage}
\centering
    \begin{figure}
    \centering
	    \includegraphics[width=90mm]{logo_lmu.jpg}
    \end{figure}
	{\scshape\LARGE Ludwigs-Maximilians Universität \par}
	\vspace{1cm}
	{\scshape\Large Skript \par}
	\vspace{1.5cm}
	{\huge\bfseries Numerik\par}
	\vspace{2cm}
	{\Large\itshape Andreas Götzfried\par}
    \vfill
	    basierend auf\par
	    Prof. Dr. Rupert \textsc{Frank}
    \vfill
	{\large \today\par}
\end{titlepage}
\tableofcontents{}

\newpage
\section{Lineare Gleichungssysteme}
\subsection{Vektor- und Matrixnorm}
    $\mathbb{K}$ steht im Folgenden immer für $\mathbb{R}\vee\mathbb{C}$.\newline
    \textbf{Definition: Norm}\newline
    Eine Norm auf den $X$ $\mathbb{K}$-Vektorraum ist eine Abbildung 
    $\norm{\cdot}:X\rightarrow[0,\infty)$, für die gilt:
    \begin{itemize}
        \item $\norm{x}_1=\sum_{x=1}^N|x_n|$
    \end{itemize}
\subsubsection{Vektornorm}

\end{document}