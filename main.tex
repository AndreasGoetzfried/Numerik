\documentclass{article}
\usepackage{graphicx}
\usepackage[utf8]{inputenc}
\usepackage{listings}
\usepackage{color}
\usepackage{amsfonts}
\usepackage{tabularx}
\usepackage{mathtools}
\usepackage[ngerman]{babel}
\newcommand\norm[1]{\left\lVert#1\right\rVert}

\begin{document}
\begin{titlepage}
\centering
    \begin{figure}
    \centering
	    \includegraphics[width=90mm]{logo_lmu.jpg}
    \end{figure}
	{\scshape\LARGE Ludwig-Maximilians Universität \par}
	\vspace{1cm}
	{\scshape\Large Skript \par}
	\vspace{1.5cm}
	{\huge\bfseries Numerik\par}
	\vspace{2cm}
	{\Large\itshape Andreas Götzfried\par}
    \vfill
	    basierend auf\par
	    Prof. Dr. Rupert \textsc{Frank}
    \vfill
	{\large \today\par}
\end{titlepage}
\tableofcontents{}

\newpage
\section{Lineare Gleichungssysteme}
\subsection{Vektor- und Matrixnorm}
    $\mathbb{K}$ steht im Folgenden immer für $\mathbb{R}\vee\mathbb{C}$.\newline
    \textbf{Definition: Norm}\newline
    Eine Norm auf den $X$ $\mathbb{K}$-Vektorraum ist eine Abbildung 
    $\norm{\cdot}:X\rightarrow[0,\infty)$, für die gilt:
    \begin{itemize}
        \item $\norm{x}=0 \Leftrightarrow x=0$
        \item $\norm{\alpha x}=|\alpha|\cdot \norm{x}$
        \item $\norm{x+y}\le \norm{x}+\norm{y}$
    \end{itemize}
\subsubsection{Vektornorm}
    Es gibt verschiedene wichtige \textbf{Vektornormen}, wie
    \begin{itemize}
        \item Betragssummennorm: $\norm{x}_1=\sum_{n=1}^N|x_n|$
        \item Euklidische Norm: $\norm{x}_2=\sqrt{\sum_{n=1}^N|x_n|^2}$
        \item Maximumsnorm: $\norm{x}_\infty=\max_{n=1,...,N}|x_n|$
    \end{itemize}
    Ein wichtiger Satz ist $\forall_{x,y\in X}|\norm{x}-\norm{y}|\le \norm{x-y}$.
\subsubsection{Matrixnorm}
    Auch existieren einige wichtige \textbf{Matrixnormen}, wie
    \begin{itemize}
        \item Spaltensummennorm: $\norm{A}_1=\max_{1\le m\le N}\sum_{n=1}^N|a_{nm}|$
        \item Zeilensummennorm: $\norm{A}_\infty=\max_{1\le n\le N}\sum_{m=1}^N|a_{nm}|$
        \item Frobenius/Hilbert-Schmidtnorm: $\norm{A}_F=\sqrt{\sum_{n,m=1}^N|a_{nm}|^2 }$
    \end{itemize}
    Die von einer Vektornorm \textbf{induzierte Matrixnorm} ist
    \begin{equation}
        \norm{A}\coloneqq \sup_{x\not= 0}\frac{\norm{Ax}}{\norm{x}}
    \end{equation}
    und, für welche die \textbf{Submultiplikativität} gilt
    \begin{equation}
        \forall_{A,B\in\mathbb{K}^{N\times N}}\norm{AB}\le\norm{A}\norm{B}
    \end{equation}
    Ein weiterer Satz behauptet, dass für die induzierten Matrixnormen $\norm{A}_2 \le \norm{A}_F \wedge \norm{A}_2 \le \sqrt{\norm{A}_2\norm{A}_\infty}$ gilt. Ebenso ist $\norm{A}_F= \sqrt{\sum_{n=1}^N\lambda_n(A^*A)}$ mit $\lambda_i$ als Eigenwerten von $A^*A$.\newline
    Sei $A^*A$ noch hermitsch und positiv definit, dann $\norm{A}_2 = \sqrt{\max_{1\le n\le N}\lambda_n(A^*A)}$. Sei zusätzlich noch $A$ hermitsch gilt $\norm{A}_2=\max_{1\le n\le N}|\lambda_n(A)|$.
\subsubsection{Konditionszahl}
    \textbf{Definition: Konditionszahl}\newline
    Die Konditionszahl einer regulären Matrix $A\in \mathbb{K}^{N\times N}$ bezüglich der einer Matrixnorm ist $\textrm{cond} A \coloneqq  \norm{A} \norm{A^{-1}}$.\newline
    Die Konditionszahl bleibt bei Multiplikation mit einem Element aus $\mathbb{K}$ gleich.\newline
    Als wichtiger Satz gilt 
    \begin{equation}
        \textrm{cond} A =\frac{\sup_{\norm{x}=1}\norm{Ax}}{\inf_{\norm{x}=1}\norm{Ax}}
    \end{equation}
    Bezüglich der euklidischen Norm kann dieser Satz noch vereinfacht werden zu $\textrm{cond} A = \sqrt{\max_n\lambda_n(A^*A)/\min_n\lambda_n(A^*A)}$. Insbesondere für $A$ hermitsch $\textrm{cond} A = \max_n\lambda_n(A)/\min_n\lambda_n(A)$.
    Ein weiterer wichtiger Satz für reguläre (invertierbare) Matrizen $A\in \mathbb{K}^{N\times N}$ ist 
    \begin{equation}
        b,x,\Delta b,\Delta x\in \mathbb{K} \wedge A\cdot x=b\wedge A(x+\Delta x)=b + \Delta b \Rightarrow 
    \end{equation}
    \begin{equation*}
        \norm{\Delta x}\le \norm{A^{-1}} \norm{\Delta b}\wedge \frac{\norm{\Delta x}}{\norm{x}} \le \textrm{cond} A \cdot  \frac{\norm{\Delta b}}{\norm{b}}
    \end{equation*}
\subsubsection{Störungsresultate für Matrizen}
    Ein sehr wichtiges Lemma zu Störungsresultaten ist folgendes. Sei $\norm{\cdot}$ induziert und $B\in \mathbb{K}^{N\times N}$ mit $\norm{B}<1$. Dann ist $\mathbb{1}+B$ regulär mit $\norm{(\mathbb{1}+B)^{-1}}\le \frac{1}{1-\norm{B}}$ \textbf{HIER FEHLT ETWAS UND MAN MUSS SCHAUEN OB 1 DIE EINHEITSMATRIX ODER DIE EINSMATRIX IST!!!!!!!}\newline
    \textbf{Definition: diagonaldominante Matrix}\newline
    Eine Matrix $A\in \mathbb{K}^{N\times N}$ heißt strikt diagonaldominant, wenn $\forall_{n=1,...,N}|a_{nn}|>\sum_{m=1,m\not = n}^N|a_{nm}|$.\newline 
    Ein wichtiges Lemma zu diagonaldominanten Matrizen $A$ ist, dass diese dann regulär sind und $\norm{A^{-1}}\le \max_{1\le n\le N}(|a_{nn}|-\sum_{m\not = n}|a_{nm}|)^{-1}$.
\subsection{LR-Zerlegung}

\newpage
\section{weitere Section}
\end{document}